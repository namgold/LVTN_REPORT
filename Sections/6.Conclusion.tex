\section{Kết quả đạt được}
Thông qua quá trình phân tích yêu cầu hệ thống, nghiên cứu các công nghệ và tiến hành làm luận văn, nhóm đã xây dựng thành công hệ thống phòng Tổ chức-Hành chính trên nền tảng Web. Hệ thống đã đáp ứng được các yêu cầu, nghiệp vụ được giao với giao diện hiện đại và dễ sử dụng với các tính năng nổi bật sau:
\begin{itemize}
    \item Trang chủ tổng hợp hình ảnh giới thiệu, sự kiện, tin tức và các thông tin liên hệ với giao diện phù hợp cho nhiều loại thiết bị và quản trị viên có thể tuỳ chỉnh các thành phần hiển thi.
    \item Xây dựng được chức năng quản lý CBCNV của trường bao gồm các chức năng cơ bản như sau:
        \subitem - Thêm cán bộ mới vào hệ thống.
        \subitem - Quản lý thông tin viên chức của cán bộ.
        \subitem - Quản lý các loại hồ sơ, văn bằng của cán bộ.
        \subitem - Xuất thông tin của cán bộ dưới theo mẫu của Bộ nội vụ dưới dạng tệp tin word.
    \item Xây dựng được các chức năng hỗ trợ quản lý các quá trình nghiệp vụ:
        \subitem - Đối với cán bộ của phòng Tổ chức - Hành chính có các chức năng như thêm, chỉnh sửa, xóa các quá trình nghiệp vụ của CBCNV.
        \subitem - Đối với CBCNV có thể xem được các quá trình nghiệp vụ của chính mình.
    \item Xây dựng các chức năng quản lý yêu cầu về quá trình nghiệp vụ của CBCNV:
        \subitem - Chức năng duyệt các yêu cầu đối với cán bộ phòng Tổ chức - Hành chính.
        \subitem - Chức năng tạo các yêu cầu mới cho các quá trình nghiệp vụ đối với CBCNV.
        \subitem - Thêm bình luận, hình ảnh minh chứng cho các yêu cầu
    \item Gửi email thông báo: Những người dùng liên quan sẽ nhận được mail thông báo trong các trường hợp sau:
        \subitem - Có yêu cầu mới được tạo.
        \subitem - Yêu cầu đã được chỉnh sửa.
        \subitem - Yêu cầu đã được duyệt/từ chối.
        \subitem - Yêu cầu có bình luận mới.
    \item Trang thống kê các số liệu của hệ thống.
    \item Hệ thống hỗ trợ tìm kiếm các bảng, trong mỗi bảng có hỗ trợ tìm kiếm theo các cột.
    \item Giảm tải được khối lượng công việc cho phòng Tổ chức - Hành chính. Trước đây, cán bộ phòng Tổ chức - Hành chính phải thực hiện tất cả công việc quản lý cán bộ. Thay vào đó, cán bộ có thể tự sử dụng hệ thống để cập nhật những thông tin của mình.
\end{itemize}
\section{Ưu điểm}
Với các chức năng đã thực hiện được, hệ thống Tổ chức - Hành chính có các ưu điểm sau:
\begin{itemize}
    \item Hệ thống đã được đưa lên máy chủ, một số chức năng đã được phòng Tổ chức - Hành chính sử dụng.
    \item Hệ thống hỗ trợ nhập liệu một cách nhanh chóng và dễ dàng.
    \item Giao diện đẹp, thân thiện với người dùng.
    \item Đơn giản hoá các quy trình nghiệp vụ rườm rà.
    \item Sử dụng được trên nhiều thiết bị.
    \item Tính bảo mật cao, phân chia chức năng theo vai trò của người dùng.
    \item Mã nguồn dưới dạng module dễ bảo trì và  phát triển thêm nghiệp vụ mới.
\end{itemize}
\section{Hạn chế}
Bên cạnh những ưu điểm nêu trên thì hệ thống còn tồn tại một vài điểm hạn chế:
\begin{itemize}
    \item Với khối lượng dữ liệu lớn, hệ thống chưa tối ưu được tốc độ xử lý.
    \item Quá trình kiểm thử được thực hiện nghiêm túc nhưng với sự phúc tạp của hệ thống không thể tránh khỏi những lỗi không mong muốn.
\end{itemize}
\section{Hướng phát triển}
Những định hướng cho hệ thống trong tương lai:
\begin{itemize}
    \item Tích hợp các cơ chế kỹ thuật khắc phục hệ thống, tự động sao lưu.
    \item Hiện thực ứng dụng trên nền tảng Mobile.
    \item Nhân rộng hệ thống cho phòng Tổ chức - Hành chính của các đơn vị, trường học khác.
\end{itemize}