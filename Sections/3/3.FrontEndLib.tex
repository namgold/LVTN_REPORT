\subsection{Các thư viện và framework cho Frontend}
Front-end của một ứng dụng được hiểu là phần tương tác trực tiếp với người dùng. Nhà phát triển sử dụng kết hợp các ngôn ngữ như HTML, CSS, JavaScipt,... thiết kế nên giao diện để người dùng có thể xem và tương tác trực tiếp với ứng dụng.

Hiện nay, các ứng dụng có nội dung ngày càng lớn, yêu cầu của người dùng về UI-UX ngày càng cao, gây khó khăn khi phát triển. Vì vậy, các front-end framework ra đời giúp tiết kiệm thời gian lập trình, tối ưu hoá và dễ dàng tạo ra các tương tác thân thiện với người dùng.

Tuy nhiên lại có rất nhiều front-end framework khác nhau, với các tính năng và đặc điểm nổi bật riêng. Vì vậy, nhóm quyết định tìm hiểu các framework phổ biến nhất để từ đó đưa ra sự lựa chọn phù hợp cho ứng dụng của mình.
\subsubsection{React}
React là một thư viện được JavaScipt để phát triển giao diện người dùng, được xây dựng cho các nhà phát triển của Facebook và Instagram.

React nổi lên với xu hướng Single Page Application, đơn giản và dễ dàng phối hợp với những thư viện Javascript khác. Trong báo cáo khảo sát lập trình viên stack-overflow 2019, React đã vươn lên vị trí thứ 2 ở mục các Web Framework phổ biến nhất.

Các đặc điểm chính:
\begin {itemize}
\item Được sử dụng để xây dựng ứng dụng đơn trang
\item React hỗ trợ việc xây dựng những thành phần (components) UI có tính tương tác cao, có trạng thái và có thể tái sử dụng.
\item React không chỉ hoạt động trên phía client, mà còn được render trên server và có thể kết nối với nhau
\end {itemize}

\subsubsection{jQuery}
jQuery là một thư viện JavaScript nhanh, nhỏ gọn và nhiều chức năng, được thiết kế để đơn giản hóa lập trình phía máy người dùng của HTML. Jquery được phát hành vào tháng 1 năm 2006 tại BarCamp NYC bởi John Resig và là thư viện JavaScript phổ biến nhất được sử dụng ngày nay.

\subsubsection{AngularJS \& Angular}
\subsubsubsection{AngularJS}
AngularJS hay Angular1 là một trong những công nghệ JavaScript phổ biến nhất trong giới phát triển Front-End. Nó được hậu thuẫn bởi Google và một cộng đồng lớn. AngularJS được tạo ra để xây dựng các ứng dụng web động (dynamic web app), và thường được sử dụng để tạo ra các ứng dụng đơn trang ( Single Page Application - SPA).

Phiên bản AngularJS được phát triển dựa trên JavaScipt nên dễ tiếp cận và được sử dụng rộng rãi. Tuy nhiên về mặt hiệu năng thì AngularJS không được đánh giá cao, thường bị đem ra so sanh với React. Hiện nay, những công ty phát triển phần mềm thường không chọn AngularJS để phát triển một sản phẩm mới.
\subsubsubsection{Angular}
Các phiên bản tiếp theo của Angular 2,4,5,6,7,8 có tên chính thức là Angular, ra đời với nhiều sự khác biệt và cải tiến so với AngularJS:
\begin {itemize}
\item TypeScript thay cho JavaScript làm ngôn ngữ mặc định
\item Kiến trúc component-based
\item Cải thiện hiệu năng trên nền tảng web và mobile
\end {itemize}
Vì Angular2 được viết lại từ đầu nên khác biệt hoàn toàn so với AngularJS nên việc nâng cấp từ AngulaJS lên Angular2 khá khó khăn. Hiện nay cộng đồng lập trình viên vẫn đang sử dụng cả 2 

\subsubsection{VueJS}
Vue.js gọi tắt là Vue, là một framework linh động (nguyên bản tiếng Anh: progressive framework) dùng để xây dựng giao diện người dùng (user interfaces). Khác với các framework nguyên khối (monolithic), Vue được thiết kế từ đầu theo hướng cho phép và khuyến khích việc phát triển ứng dụng theo từng bước. Khi phát triển lớp giao diện (view layer), người dùng chỉ cần dùng thư viện lõi (core library) của Vue, vốn rất dễ học và tích hợp với các thư viện hoặc dự án có sẵn. Cùng lúc đó, nếu kết hợp với những kĩ thuật hiện đại như SFC (single file components) và các thư viện hỗ trợ, Vue cũng đáp ứng được dễ dàng nhu cầu xây dựng những ứng dụng một trang (SPA - Single-Page Applications) với độ phức tạp cao hơn nhiều.

Vue có nhiều điểm tương đồng với React với Virtual DOM và các component có thể tái sử dụng.Vue.js cũng hỗ trợ tích hợp những thư viện khác vào framework mà không cần tốn quá nhiều công sức.
\subsubsection{Backbone.js}
Backbone là một thư viện JavaScript, phục vụ cho việc phát triển frontend. Backbone sử dụng các component Event, Model, Collection, View, Router để tạo nên ứng dụng web. Backbone được các lập trình viên dùng nhiều bởi nó dễ sử dụng và áp dụng cho các ứng dụng JavaScript.

Backbone đơn giản, gọn nhẹ và có thể kết hợp cùng các library khác. Các ứng dụng thực tế sử dụng Backbone như LinkedIn Mobile, Foursquare, Basecamp,...

Backbone không có Controller nên model và view được đồng bộ rất tốt. Tuy nhiên, việc không có Controller đôi khi có thể gây ra hiện tượng nhập nhằn vì các thao tác làm trong controler giờ được trải rải rác vào model và view.
\subsubsection{Ember.js}
Ember là một front-end framework JavaScript mã nguồn mở vận hành trên mô hình Model view viewmodel (MVVM). Ember cho phép phát triển tạo ra các ứng dụng đơn trang (SPA) có thể mở rộng bằng cách kết hợp các thành ngữ phổ biến và các thực tiễn tốt nhất vào khung.\\
ạkjakadjkad

\subsubsection{Kết luận và lựa chọn}

