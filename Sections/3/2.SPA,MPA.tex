\subsection{Ứng dụng đa trang và ứng dụng đơn trang}
Khi phát triển một ứng dụng, có hai kiểu thiết kế phổ biến hiện nay là ứng dụng đa trang (Multi-page Application-MPA) và ứng dụng đơn trang (Single-page Application-SPA). Mỗi kiểu thiết kế có những ưu và nhược điểm riêng phù hợp với các ứng dụng khác nhau. Vì vậy để có thể phát triển ứng dụng một cách hiệu quả, nhà phát triển phải cân nhắc lựa chọn cách thiết kế phù hợp nhất với nhu cầu của ứng dụng của mình.
\subsubsection{Ứng dụng đa trang (Multi-page Application-MPA)}
Ứng dụng đa trang chứa nhiều trang liên kết và các trang con, được điều hướng bằng menu. Ứng dụng đa trang phù hợp với hầu hết các dự án. Ứng dụng đa trang được sử dụng rộng rãi trong nhiều lĩnh vực như thương mại điện tử (amazon.com), e-learning (lynda.com),...\\\\\textit{Ưu điểm:}
\begin{itemize}
    \item Cho phép khả năng mở rộng ứng dụng thông qua menu và thanh tìm kiếm.
    \item Luồng điều hướng dễ theo dõi. 
    \item Hỗ trợ tốt cho SEO.
\end {itemize}
\textit{Nhược điểm:}
\begin{itemize}
    \item Các ứng dụng có nhiều nội dung lớn thường tải chậm, ảnh hưởng đến trải nghiệm của khách hàng.
    \item Khó thích nghi tốt với thiết bị di động.
\end {itemize}
\subsubsection{Ứng dụng đơn trang (Single-page Application-SPA):}
Ứng dụng đơn trang (SPA) là một web-app hay một website tương tác với người dùng bằng cách tải lại một phần của trang hiện tại thay vì tải toàn bộ trang mới từ máy chủ. 

Phương pháp này hạn chế sự gián đoạn trải nghiệm của người dùng khi chuyển giữa các trang, giúp ứng dụng cho trải nghiệm gần giống với desktop application.

Trong ứng dụng đơn trang, tất cả các tài nguyên cần thiết như mã HTML, JavaScript, CSS được tải duy nhất ở lần đầu tiên 


Ứng dụng đơn trang chỉ chứa 1 trang HTML, không có trang bổ sung, chẳng hạng như giới thiệu, liên hệ,...Ứng dụng đơn trang không yêu cầu tải lại trang khi sử dụng, các nội dung sẽ được tải bằng Javascript.\\\\
\textit{Ưu điểm:}
\begin{itemize}
    \item Tốc độ nhanh, do các tài nguyên được tải xuống khi trong suốt quá trình sử dụng.
    \item Có khả năng làm việc với cache tốt, nên hiệu quả khi sử dụng ở chế độ offline.
    \item Thích nghi tốt với thiết bị di động.
\end{itemize}
\textit{Nhược điểm:}
\begin {itemize}
    \item Không tốt cho SEO (các công cụ tìm kiếm). Việc tìm kiếm các nội dung trong ứng dụng đơn trang bằng các công cụ tìm kiếm (google.com, bing.com) thường ít hiệu quả.
    \item Khó khăn trong việc mở rộng.
\end {itemize}
\subsubsection{Kết luận và lựa chọn:}
Nhận thấy hệ thống phòng tổ chức hành chính có những đặc điểm như sau:
\begin {itemize}
    \item Có nhiều nội dung tương đối giống nhau nên việc sử dụng ứng dụng đa trang là không cần thiết.
    \item Hệ thống tập trung vào tác vụ xử lý nên ưu tiên tốc độ cao.
    \item Hệ thống phát triển hỗ trợ hoạt động của phòng tổ chức hành chính do đó không cần quá chú trọng vào SEO.
\end {itemize}

Vì vậy nhóm quyết định sử dụng ứng dụng đơn trang vào ứng dụng phòng tổ chức hành chính.




