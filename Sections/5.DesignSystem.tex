\section{Kiến trúc hệ thống}
\indent Hệ thống được xây dựng theo mô hình MVC kết hợp với REST API (RESTful):
\subsection{Giới thiệu REST API:}
REST (REpresentational State Transfer) là một tiêu chuẩn thiết kế API (API architectural style) cho các ứng dụng. REST được giới thiệu lần đầu tiên bởi Roy Fielding vào năm 2000 trong luận án nổi tiếng của ông ``Architectural Styles and
the Design of Network-based Software Architecture'' đại học California.

Mặc dù REST có thể sử dụng trên hầu hết mọi giao thức ứng dụng phần mềm, nhưng REST thường biết đến rộng rãi trong các giao thức của ứng dụng Web vì tận dụng được những lợi thế của HTTP, giúp nhà phát triển không cần cài đặt thêm phần mềm hoặc thư viện bổ sung để sử dụng REST API.

Một số đặc điểm cơ bản của REST:
\begin{itemize}
    \item \textbf{Uniform interface:} Các API được thiết kế một cách nhất quán theo các nguyên tắc chung. Ví dụ luôn luôn sử dụng danh từ số nhiều thay vì khi số nhiều, khi số ít, sử dụng dấu gạch ngang để phân tách giữa các từ. Tài nguyên của hệ thống phải tuân theo các quy tắt đặt tên, định dạng dữ liệu (JSON hay XML), được truy cập thông qua một cách tiếp cận phổ biến như HTTP GET và được sửa đổi thông qua các phương thức tương tự.
    \item \textbf{Client-Server:} Client và Server không cần phụ thuộc vào nhau, có thể phát triển độc lập, thậm chí sửa đổi, thay thế miễn sao interface giao tiếp giữa chúng vẫn được giữ nguyên.
    \item \textbf{Stateless}: Server không lưu trữ bất kì thông tin gì về yêu cầu của client, nó sẽ coi mọi yêu cầu là mới. Không section, không history. Vì vậy, mỗi yêu cầu từ client tới server phải chứa tất cả các thông tin cần thiết để  server hiểu được yêu cầu đó.
    \item \textbf{Code on demand:} Sử dụng HTTP status code khi có thể
\end{itemize}


Những lợi ích khi sử dụng REST API:

- Nhờ tính độc lập giữa client và server, nhà phát triển có thể dễ dàng phát triển, sửa đổi thậm chí thay thế client và server miễn sao vẫn giữ nguyên interface giao tiếp giữa chúng.

- API REST luôn độc lập với loại nền tảng hoặc ngôn ngữ: API REST luôn thích nghi với loại cú pháp hoặc nền tảng đang được sử dụng, điều này mang lại sự tự do đáng kể khi thay đổi hoặc kiểm tra môi trường mới trong quá trình phát triển. Với API REST, bạn có thể có máy chủ với các ngôn ngữ như PHP, Java, Python hoặc Node.js.

- Tính nhất quán: Các API được thiết kế một cách nhất quán giúp cho việc phát triển, bảo trì đơn giản hơn, đặc biệt phù hợp với các hệ thống có nhiều module.

Vận dụng kiến trúc REST API trong hệ thống:
Các API trong hệ thống Tổ chức-Hành chính được thiết kế theo các nguyên tắc chung, nhất quán giữa các module.
\begin{itemize}
    \item API được đặt tên theo quy định chung, phù hợp với công dụng, quyền của API, thống nhất giữa các module.
    \begin{itemize}
        \item  ``/api/tchc/qua-trinh/dao-tao/all'', ``/api/tchc/qua-trinh/khen-thuong/all'': API lấy tất cả dữ liệu của một bảng quá trình (đào tạo, khen thưởng). Ta có thể thấy các API có cấu trúc thống nhất với nhau giữa các bảng khác.
        \item  ``/api/tchc/qua-trinh/dao-tao/item/:id'', ``/api/user/qua-trinh/dao-tao/item/:id'': API lấy dữ liệu của bảng quá trình đào tạo theo ID. API thứ nhất dùng cho cán bộ phòng tổ chức hành chính, API thứ hai dành cho người dùng của hệ thống.
    \end{itemize}
    \item Sử dụng các phương thức HTTP (GET, POST, PUT, DELETE) để truy cập và xử lý dữ liệu.
    \begin{itemize}
        \item  app.post(/api/tchc/qua-trinh/dao-tao/all): cách dùng API để lấy dữ liệu.
        \item  app.post(/api/tchc/qua-trinh/dao-tao): cách dùng API để thêm dữ liệu.
        \item  app.put(/api/tchc/qua-trinh/dao-tao): cách dùng API để sửa đổi dữ liệu.
        \item  app.delete(/api/tchc/qua-trinh/dao-tao): cách dùng API để xoá dữ liệu.
    \end{itemize}
    \item Dữ liệu được server trả về theo định dạng JSON.
    \item Server trả về các HTTP code phù hợp.
    \begin{itemize}
        \item  res.status(400).send('Invalid parameter'): Trả về lỗi này khi client gửi yêu cầu sai cú pháp.
        \item  res.status(403).send('can not access data'): Trả về lỗi này khi client không có quyền truy cập dữ liệu.
        \item  res.status(404).send('Not found'): Trả về lỗi này khi server không tìm thấy dữ liệu client yêu cầu.
    \end{itemize}
\end{itemize}
\section{Mô hình Model-View-Controller:}
Nhận thấy mô hình MVC có nhiều ưu điểm và phù hợp với hệ thống (trình bày ở mục 2.1 Mô hình Model-View-Controller) nên nhóm đã sử dụng mô hình này trong thiết kế kiến trúc hệ thống.

Vận dụng mô hình MVC trong hệ thống phòng Tổ chức-Hành chính:
\begin{center}
  \captionsetup{type=figure}
  \includegraphics[width=15cm]{img/MVCInTchc.png}
  \captionof{figure}{Kiến trúc tổng thể của hệ thống}
\end{center}

\section{Nguyên tắc thiết kế hệ thống}
\indent Với một hệ thống khá phức tạp như trên cần phải có những nguyên tắc thiết kế chung
\subsection{Cấu trúc cây thư mục của mã nguồn}
\indent Dựa vào kiến trúc hệ thống đã thiết ở mục 4.2 ``hình 4.1: Kiến trúc tổng thể của hệ thống''. Nhóm đã thiết kế 

Việc cần làm đầu tiên và quan trọng nhất là tổ chức files và các thư mục. Với một thiết kế tốt thì khi làm việc chung giữa nhiều người sẽ ít bị đụng độ. Nhóm lựa chọn cách chia mỗi thành phần của trang ra từng module tách biệt với nhau, thuận lợi cho việc mở rộng và bảo trì, bảo dưỡng. Bên dưới là cấu trúc mã nguồn mà nhóm đã lựa chọn:
\begin{center}
  \captionsetup{type=figure}
  \includegraphics[width=15cm]{img/tree.png}
  \captionof{figure}{Cây thư mục của hệ thống}
\end{center}
\section{Thiết kế đối tượng người dùng}
\subsection{Danh sách đối tượng người dùng}
\begin{table}[H]
    \centering
	\begin{tabular}{|p{1cm}|p{4cm}|p{10cm}|}
    \hline
    \textbf{STT}&\textbf{Người dùng}&\textbf{Đặc tả}\\
	\hline
    1&Quản trị hệ thống&Là những người tạo ra hệ thống ban đầu, có các chức năng hỗ trợ tạo ra các cấu hình cho hệ thống, bao gồm các thao tác như: cấu hình các thông tin chung của phòng Tổ chức hành chính, cấu hình về giao diện, quản lý tài khoản người dùng\\
	\hline
    2&Trưởng phòng Tổ chức - Hành chính&Chức năng chính là chịu trách nhiệm phân công công việc cho các cán bộ thuộc phòng Tổ chức - Hành chính. Bên cạnh đó những người này còn có chức năng quản lý thông tin của cán bộ nhà trường\\
	\hline
    3&Cán bộ phòng Tổ chức - Hành chính&Là những cán bộ thuộc phòng Tổ chức - Hành chính. Họ có chức năng quản lý thông tin của toàn bộ cán bộ, quản lý thông tin các danh mục , quản lý thông tin các quá trình nghiệp vụ của nhà trường.\\
	\hline
	4&Người dùng (giảng viên, cán bộ nhà trường)&Người dùng với mục đích quản lý được thông tin, các quá trình nghiệp vụ của bản thân.\\
	\hline
\end{tabular}
\caption{Danh sách đối tượng người dùng của hệ thống}
\end{table}
\subsection{Đối tượng: Quản trị hệ thống}
\textbf{Lược đồ Usecase tổng thể}
\begin{center}
  \captionsetup{type=figure}
  \includegraphics[scale=0.8]{img/UML/Admin/adminUsecase.png}
  \captionof{figure}{Lược đồ usecase của đối tượng quản trị hệ thống}
\end{center}

\indent Đối với đối tượng là quản trị hệ thống, sau khi xác thực thành công sẽ cónhững nhóm chức năng: quản lý tài khoản và quản lý cấu hình.Với mỗi nhóm chức năng sẽ có những chức năng tương ứng được thể hiện trong các lược đồ Usecase sau:

\begin{center}
  \captionsetup{type=figure}
  \includegraphics[scale=0.5]{img/UML/Admin/quanlytaikhoan.png}
  \captionof{figure}{Lược đồ Usecase cho nhóm chức năng quản lý tài khoản}
\end{center}

Trong nhóm quản lý tài khoản bao gồm các chức năng: quản lý tài khoản người dùng và quản lý quyền của người dùng.
\begin{itemize}
    \item Quản lý tài khoản người dùng:
        \subitem Người quản trị có thể thay các tài khoản với các quyền đặc biệt cho trưởng phòng Tổ chức - Hành chính, cán bộ phòng Tổ chức - Hành chính
    \item Quản lý quyền của người dùng:
        \subitem Người quản trị hệ thống có thể thêm một quyền mới vào hệ thống, chỉnh sửa quyền, xóa quyền, cũng như gắn quyền cho người dùng nhất định.
\end{itemize}

\begin{center}
  \captionsetup{type=figure}
  \includegraphics[scale=0.4]{img/UML/Admin/quanlycauhinh.png}
  \captionof{figure}{Lược đồ Usecase cho nhóm chức năng quản lý cấu hình}
\end{center}

Nhóm quản lý cấu hình gồm các chức năng: cấu hình giao diện trang chủ, cấu hình thông tin trang web, cấu hình email cho hệ thống, cấu hình thông tin danh mục.

- Cấu hình giao diện trang chủ: Tùy chỉnh các thành phần giao diện, lựa chọn các thành phần giao diện sẽ xuất hiện ở trang chủ.

- Cấu hình thông tin trang web: Cấu hình các thông tin chung cho hệ thống, thông tin liên lạc của phòng Tổ chức - Hành chính.

- Cấu hình email hệ thống: Tủy chỉnh các mẫu email tự động của hệ thống.

- Cấu hình thông tin các danh mục: Xây dựng thông tin những danh mục mặc định phục vụ cho việc nhập và lữu trữ dữ liệu trong hệ thống.
\subsection{Đối tượng: Trường phòng Tổ chức - Hành chính và Cán bộ phòng Tổ chức - Hành chính}
\textbf{Lược đồ Usecase tổng thể}
\begin{center}
  \captionsetup{type=figure}
  \includegraphics[scale=0.5]{img/UML/TchcStaff/Admin.png}
  \captionof{figure}{Lược đồ Usecase cho đối tượng Trường phòng và cán bộ phòng Tổ chức - Hành chính}
\end{center}

Sau khi đăng nhập vào hệ thống, người dùng là cán bộ phòng Tổ chức - Hành chính sẽ có các nhóm chức năng sau: quản lý cán bộ, quản lý các quá trình nghiệp vụ, quản lý các tác vụ, quản lý hồ sơ của cán bộ.

Với mỗi nhóm chức năng trên có lược đồ Usecase dưới đây:
\begin{center}
  \captionsetup{type=figure}
  \includegraphics[scale=0.6]{img/UML/TchcStaff/quanlycanbo.png}
  \captionof{figure}{Lược đồ Usecase nhóm chức năng quản lý cán bộ }
\end{center}

\begin{center}
  \captionsetup{type=figure}
  \includegraphics[scale=0.5]{img/UML/TchcStaff/quanlyquatrinh.png}
  \captionof{figure}{Lược đồ Usecase nhóm chức năng quản lý quá trình nghiệp vụ}
\end{center}

\begin{center}
  \captionsetup{type=figure}
  \includegraphics[scale=0.5]{img/UML/TchcStaff/quanlytacvu.png}
  \captionof{figure}{Lược đồ Usecase nhóm chức năng quản lý tác vụ nghiệp vụ}
\end{center}

\begin{center}
  \captionsetup{type=figure}
  \includegraphics[scale=0.5]{img/UML/TchcStaff/quanlyhoso.png}
  \captionof{figure}{Lược đồ Usecase cho chức năng quản lý hồ sơ }
\end{center}

Đối với đối tượng là Trưởng phòng Tổ chức - Hành chính sau khi đăng nhập sẽ có tất cả các nhóm chức năng như trên của cán bộ của phòng Tổ chức - Hành chính. Bên cạnh đó đối tượng trưởng phòng Tổ chức - Hành chính còn có thêm chức năng chỉ định cán bộ của phòng Tổ chức - Hành chính để duyệt các tác vụ chưa duyệt.

\subsection{Đối tượng: Cán bộ của trường}
\textbf{Lược đồ Usecase tổng thể}
\begin{center}
  \captionsetup{type=figure}
  \includegraphics[scale=0.5]{img/UML/User/user.png}
  \captionof{figure}{Lược đồ Usecase đối tượng cán bộ của trường}
\end{center}
Đối với người dùng là cán bộ của trường, sau khi đăng nhập sẽ có chức năng quản lý thông tin của cá nhân, quản lý các quá trình nghiệp vụ của bản thân.
Mỗi nhóm chức năng có lược đồ usecase dưới đây
\begin{center}
  \captionsetup{type=figure}
  \includegraphics[scale=0.4]{img/UML/User/quanlythongtin.png}
  \captionof{figure}{Lược đồ Usecase nhóm chức năng quản lý thông tin tài khoản}
\end{center}

\begin{center}
  \captionsetup{type=figure}
  \includegraphics[scale=0.3]{img/UML/User/QuanLyQuaTrinhTacVu.png}
  \captionof{figure}{Lược đồ Usecase nhóm chức năng quản lý quá trình nghiệp vụ của bản thân}
\end{center}