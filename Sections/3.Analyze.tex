\section{Phân quyền người dùng trong hệ thống}
Dựa vào những yêu cầu thực tế của hệ thống và những khảo sát thực tế đối với các cá nhân có nhu cầu sử dụng, xây dựng hệ thống Tổ chức - Hành chính. Nhóm đưa ra các vai trò, người dùng chính trong hệ thống như sau:
\begin{itemize}
    \item Quản trị hệ thống
    \item Cán bộ quản lý phòng Tổ chức - Hành chính
    \item Cán bộ phòng Tổ chức - Hành chính
    \item Cán bộ, nhân viên, giáo viên của trường
\end{itemize}
\section{Những chức năng mà hệ thống đảm nhiệm}
\begin{itemize}
    \item Quản lý thông tin cán bộ công nhân viên nhà trường
    \begin{itemize}
        \item Tính năng tìm kiếm, thêm, xoá, sửa, lưu thông tin.
        \item Tính năng nhập dữ liệu từ file Excel.
        \item Quản lý lý lịch, thời gian công tác, ngày nghỉ, trình độ.
        \item Tính năng truy vấn dữ liệu, xuất dữ liệu theo yêu cầu. 
    \end{itemize}
    \item Website truyền thông với các tin tức mới nhất, giao diện đẹp, dễ sử dụng.
    \begin{itemize}
        \item Thực hiện thiết kế UI, UX hợp lý.
        \item Có công cụ để viết bài, tin tức mới. Công cụ này phải thân thiện, dễ sử dụng.
        \item Có công cụ liên kết với các phòng ban khác, nhận dữ liệu tin tức từ các trang phòng ban khác của trường.
    \end{itemize}
    \item Các tính năng thống kê như tính điểm thi đua, điểm thưởng,...
    \begin{itemize}
        \item Hiện thực các công cụ thống kê.
        \item Hỗ trợ tính các tính năng chuyên môn như tính điểm thi đua, tính điểm thưởng.
    \end{itemize}
    \item Quản lý bảo hiểm cho cán bộ.
    \begin{itemize}
        \item Thực hiện lưu trữ, cập nhật dữ liệu bảo hiểm cho cán bộ.
        \item Có công cụ nhắc nhở những cán bộ sắp hết hạn bảo hiểm.
        \item Có công cụ để truy xuất lịch sử mua bảo hiểm của cán bộ.
    \end{itemize}
    \item Quản lý các cấp, đơn vị hành chính của nhà trường.
    \begin{itemize}
        \item Thực hiện lưu trữ các cấp, đơn vị hành chính hiện tại của nhà trường.
        \item Có công cụ để thực hiện cập nhật, thêm mới, xoá các đơn vị.
        \item Có công cụ để truy xuất các đơn vị hiện tại của nhà trường.
    \end{itemize}
    \item Quản lý quy trình khen thưởng.
    \begin{itemize}
        \item Quản lý các huân chương, bằng khen của cán bộ nhân viên.
        \item Quản lý các cấp khen thưởng như cấp trường, Thành phố, cấp Bộ \& nhà nước,...
        \item Quản lý thời gian khen thưởng.
    \end{itemize}
    \item Quản lý quy trình kỷ luật.
    \begin{itemize}
        \item Quản lý các hình thức kỷ luật, lý do kỷ luật của cán bộ nhân viên.
        \item Quản lý các cấp kỷ luật như cấp trường, Đảng,...
        \item Quản lý ghi chú khiếu nại về kỷ luật.
    \end{itemize}
    \item Quản lý lương cán bộ.
    \begin{itemize}
        \item Quản lý thông tin lương của các cán bộ nhân viên nhà trường gồm có hệ số lương, bậc, tỷ lệ, thời gian hưởng.
        % \item Tính toán lương cho cán bộ, công nhân viên.
    \end{itemize}
    \item Quản lý chức danh, chức vụ của cán bộ.
    \begin{itemize}
        \item Quản lý thông tin chức danh, chức vụ của cán bộ
        \item Quản lý thông tin bổ nhiệm chức danh, chức vụ cho cán bộ
    \end{itemize}
    \item Bồi dưỡng nghiệp vụ cho cán bộ.
    \begin{itemize}
        \item Quản lý quy trình bồi dưỡng nghiệp vụ cho cán bộ, với nội dung phong phú như Tin học ứng dụng, Anh văn, Lý luận chính trị, Quản lý giáo dục,...
        \item Quy trình bồi dưỡng tổ chức ở nhiều địa điểm như Học viên Hành chính Quốc gia, Học viện Chính trị Quốc gia, Đại học tổng hợp,...
        \item Quản lý chứng chỉ, bằng tốt nghiệp của các khóa bồi dưỡng, đào tạo.
    \end{itemize}
    \item Quản lý cán bộ công tác trong và ngoài nước.
    \begin{itemize}
        \item Quản lý quy trình công tác cho cán bộ
        \item Quy trình công tác với nhiều mục đích như dự hội nghị, hội thảo, học tiến sĩ, học thạc sĩ,...
        \item Quy trình công tác được tổ chức ở nhiều nơi trong nước cũng như quốc tế
    \end{itemize}
    \item Tuyển chọn cán bộ
    \begin{itemize}
        \item Quản lý thông báo tin tuyển dụng cán bộ
        \item Quản lý thông tin của các ứng viên nộp hồ sơ
        \item Sắp xếp lịch thi công chức
    \end{itemize}
    \item Chia sẻ thông tin cho các cán bộ, phòng ban khác.
    \begin{itemize}
        \item Hệ thống có khả năng liên kết với các hệ thống khác trong trường như hệ thống phòng Đào tạo, phòng Công tác Chính trị, các khoa và phòng ban trong trường.
        \item Có khả năng chia sẻ các dữ liệu cần thiết với các hệ thống khác, kết hợp, trở thành một bộ phận, góp phần xây dựng nên hệ thống quản lý toàn trường hoàn chỉnh.
    \end{itemize}
    \item Quản lý lương, thưởng và các kế hoạch bảo hiểm, trợ cấp của người lao động
    \begin{itemize}
        \item Quản lý hệ số lương, mức lương của từng cán bộ
        \item Quản lý mức thưởng với các mục đích khen thưởng kèm theo
        \item Quản lý mức phụ cấp cho cán bộ
        \item Quản lý quy trình đóng bảo hiểm xã hội
    \end{itemize}
    \item Xuất các báo cáo, thống kê theo yêu cầu và theo định kỳ
    \begin{itemize}
        \item Hệ thống có khả năng xuất các báo cáo, thống kê theo yêu cầu.
        \item Giao diện thống kê trực quan
    \end{itemize}
\end{itemize}
