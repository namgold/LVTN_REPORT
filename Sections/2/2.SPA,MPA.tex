\section{Ứng dụng đa trang và ứng dụng đơn trang}
Khi phát triển một ứng dụng, có hai kiểu thiết kế phổ biến hiện nay là ứng dụng đa trang (Multi-page Application-MPA) và ứng dụng đơn trang (Single-page Application-SPA). Mỗi kiểu thiết kế có những ưu và nhược điểm riêng phù hợp với các ứng dụng khác nhau. Vì vậy để có thể phát triển ứng dụng một cách hiệu quả, nhà phát triển phải cân nhắc lựa chọn cách thiết kế phù hợp nhất với nhu cầu ứng dụng của mình.
\subsection{Ứng dụng đa trang (Multi-page Application-MPA)}
\textbf{Giới thiệu:}

Ứng dụng đa trang (MPA) là một web-app hay một website chứa nhiều trang liên kết và các trang con, được điều hướng bằng menu. Ứng dụng đa trang hoạt động theo kiểu "truyền thống", các thay đổi như hiển thị dữ liệu sẽ được thực hiện bằng cách hiển thị một trang mới trong trình duyệt.

Ứng dụng đa trang phù hợp với hầu hết các dự án. Hiện nay, ứng dụng đa trang được sử dụng rộng rãi trong nhiều lĩnh vực như thương mại điện tử (amazon.com), e-learning (lynda.com),...

\textbf{Ưu điểm:}
\begin{itemize}
    \item Cho phép khả năng mở rộng ứng dụng không giới hạn thông qua menu.
    \item Luồng điều hướng dễ dàng theo dõi. 
    \item Hỗ trợ tốt cho SEO, các ứng dụng đa trang dễ dàng phân cấp, sắp xếp các từ khoá cho từng trang, từng sản phẩm.
\end {itemize}

\textbf{Nhược điểm:}

\begin{itemize}
    \item Các ứng dụng đa trang có nhiều nội dung lớn thường tải chậm, ảnh hưởng đến trải nghiệm của người dùng.
    \item Khó thích nghi tốt với thiết bị di động.
\end {itemize}
\subsection{Ứng dụng đơn trang (Single-page Application-SPA):}
\textbf{Giới thiệu:}

Ứng dụng đơn trang (SPA) là một web-app hay một website tương tác với người dùng bằng cách tải lại một phần của trang hiện tại thay vì tải toàn bộ trang mới từ máy chủ. Cách hoạt động này hạn chế sự gián đoạn trải nghiệm của người dùng khi chuyển giữa các trang, giúp cho việc trải nghiệm ứng dụng gần giống với các desktop application.

Trong ứng dụng đơn trang, tất cả các tài nguyên cần thiết như mã HTML, JavaScript, CSS được tải duy nhất ở lần đầu tiên mở trang web. Các tài nguyên khác sẽ được tự động tải và thêm vào trang khi cần, thường là để đáp ứng hành động của người dùng. Ứng dụng SPA chỉ tải duy nhất một lần trong suốt quá trình sử dụng vì vậy giúp tiết kiệm băng thông cũng như giảm thời gian chờ đợi.

Ngày nay, ứng dụng đơn trang được sử dụng rộng rãi trên toàn thế giới, đặc biệt phù hợp các ứng dụng với lượng truy cập lớn, yêu cầu tốc độ cao như mạng xã hội, email, map. Một số ứng dụng sử dụng SPA nổi tiếng như Facebook, Instagram, Google Map, Gmail,...

\textbf{Ưu điểm:}
\begin{itemize}
    \item Tốc độ nhanh,
    \item Có khả năng làm việc với cache tốt, nên hiệu quả khi sử dụng ở chế độ offline.
    \item Thích nghi tốt với thiết bị di động.
\end{itemize}

\textbf{Nhược điểm:}

\begin {itemize}
    \item Không tối ưu hoá cho SEO tốt (các công cụ tìm kiếm). Việc tìm kiếm các nội dung trong ứng dụng đơn trang bằng các công cụ tìm kiếm (google.com, bing.com) thường ít hiệu quả.
    \item Khó khăn trong việc mở rộng.
\end {itemize}
\subsection{Kết luận và lựa chọn}
Nhận thấy hệ thống phòng tổ chức hành chính có những đặc điểm như sau:
\begin {itemize}
    \item Có nhiều nội dung tương đối giống nhau nên việc sử dụng ứng dụng đa trang là không cần thiết.
    \item Hệ thống tập trung vào tác vụ xử lý nên ưu tiên tốc độ cao.
    \item Hệ thống phát triển hỗ trợ hoạt động của phòng tổ chức hành chính do đó không cần quá chú trọng vào SEO.
\end {itemize}

Vì vậy nhóm quyết định sử dụng ứng dụng đơn trang vào ứng dụng phòng tổ chức hành chính.




