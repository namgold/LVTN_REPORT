\section{Các nền tảng Frontend}
Front-end của một ứng dụng được hiểu là phần tương tác trực tiếp với người dùng. Nhà phát triển sử dụng kết hợp các ngôn ngữ như HTML, CSS, Javascipt,... thiết kế nên giao diện để người dùng có thể xem và tương tác trực tiếp với ứng dụng.

Hiện nay, các ứng dụng có nội dung ngày càng lớn, yêu cầu về giao diện và trải nghiệm người dùng ngày càng cao, gây khó khăn khi phát triển. Vì vậy, các front-end framework và thư viện ra đời giúp tiết kiệm thời gian lập trình, tối ưu hoá và dễ dàng tạo ra các tương tác thân thiện với người dùng.

Tuy nhiên lại có rất nhiều front-end framework và thư viện khác nhau, với các tính năng và đặc điểm nổi bật riêng. Vì vậy, nhóm quyết định tìm hiểu các thư viện, framework phổ biến nhất để từ đó đưa ra sự lựa chọn phù hợp cho ứng dụng của mình.
\subsection{React}
\begin{center}
  \captionsetup{type=figure}
  \includegraphics[width=4cm]{img/React_logo.png}
  \captionof{figure}{React}
\end{center}
%  https://skillcrush.com/2019/05/14/what-is-react-js/
React là một thư viện Javascipt mã nguồn mở, được sử dụng để xây dựng giao diện người dùng, được tạo ra bởi các nhà phát triển của Facebook.

React nổi lên với xu hướng ứng dụng đơn trang, đơn giản và dễ dàng phối hợp với những thư viện Javascript khác. Trong báo cáo khảo sát lập trình viên StackOverflow 2019, React đã vươn lên vị trí thứ 2 ở mục các web framework phổ biến nhất.

Ngoài việc cung cấp mã thư viện React có thể sử dụng lại (tiết kiệm thời gian phát triển và giảm khả năng xảy ra lỗi mã hóa), React còn đi kèm với hai tính năng chính làm tăng thêm sức hấp dẫn cho các nhà phát triển JavaScript:
\begin{itemize}
    \item JSX
    \item Virtual DOM
\end{itemize}

Nhưng trên hết, React JS là một dự án nguồn mở, có nghĩa là bất kỳ ai cũng có thể tải xuống và sửa đổi mã nguồn miễn phí. Điều này cũng có nghĩa là, bất kỳ chức năng UI cụ thể nào mà bạn mong muốn giải quyết với React JS, đều có thư viện React để đáp ứng nhu cầu của bạn. Kích thước thư viện React của bạn có thể tăng theo cấp số nhân với các tiện ích thư viện được quản lý bởi cộng đồng React, bao gồm từ các bộ sưu tập các tính năng UI cá nhân đến các mẫu React JS hoàn chỉnh để xây dựng UI từ đầu.

\textbf{JSX:}

Phần chính của bất kỳ trang web cơ bản nào là HTML. Trình duyệt web đọc các tài liệu này và hiển thị chúng trên máy tính, máy tính bảng hoặc điện thoại dưới dạng trang web. Trong quá trình này, các trình duyệt tạo ra một thứ gọi là Mô hình Đối tượng Tài liệu (Document Object Model - DOM). Sau đó, các nhà phát triển có thể thêm nội dung động vào các dự án của họ bằng cách sửa đổi DOM bằng các ngôn ngữ như JavaScript.

JSX (viết tắt của JavaScript eXtension) là một phần mở rộng của React giúp các nhà phát triển web dễ dàng sửa đổi DOM của họ bằng cách sử dụng mã kiểu HTML đơn giản. Và kể từ khi hỗ trợ trình duyệt React JS mở rộng cho tất cả các trình duyệt web hiện đại, thì JS JSX tương thích với mọi nền tảng trình duyệt mà ta có thể đang làm việc.

\textbf{Virtual DOM:}

Trong trường hợp không sử dụng React JS (và JSX), trang web sẽ sử dụng HTML để cập nhật DOM của nó (quá trình khiến mọi thứ thay đổi trên màn hình mà không cần người dùng phải tự làm mới trang). Điều này hoạt động tốt đối với các trang web đơn giản, tĩnh, nhưng đối với các trang web động có sự tương tác của người dùng nặng thì nó có thể trở thành một vấn đề (vì toàn bộ DOM cần tải lại mỗi khi người dùng nhấp vào tính năng gọi để làm mới trang).

Tuy nhiên, nếu nhà phát triển sử dụng JSX để thao tác và cập nhật DOM của nó, React JS sẽ tạo ra thứ gọi là Virtual DOM. Virtual DOM (DOM ảo) là một bản sao của DOM, và React JS sử dụng bản sao này để xem những phần nào của DOM thực tế cần thay đổi khi xảy ra sự kiện (như người dùng nhấp vào nút).

Giả sử rằng một người dùng nhập một bình luận cho một bài đăng trên blog và nhấn nút Bình luận trực tuyến. Nếu không sử dụng React JS, toàn bộ DOM sẽ phải cập nhật để phản ánh sự thay đổi này (sử dụng thời gian và sức mạnh xử lý để thực hiện cập nhật này). Mặt khác, React quét Virtual DOM để xem những gì đã thay đổi sau hành động của người dùng (trong trường hợp này là một bình luận được thêm vào) và chỉ cập nhật có chọn lọc phần đó của DOM.

Kiểu cập nhật có chọn lọc này tốn ít sức mạnh tính toán hơn và thời gian tải ít hơn, nghe có vẻ không tiết kiệm được nhiều khi ta nói về một bình luận blog, nhưng khi nói về một trang web tất cả đều động và cập nhật gắn với thậm chí là trang web hơi phức tạp hơn thì ta sẽ nhận ra rằng nó tiết kiệm hơn rất nhiều.

\textbf{Đặc điểm và thế mạnh:}
\begin {itemize}
\item Được sử dụng để xây dựng ứng dụng đơn trang.
\item React cho phép các nhà phát triển tạo ra các ứng dụng web lớn có thể thay đổi dữ liệu mà không cần tải lại trang.
\item React hỗ trợ việc xây dựng những thành phần (components) giao diện người dùng có tính tương tác cao, có trạng thái và có thể tái sử dụng.
\item React không chỉ hoạt động trên phía người dùng, mà còn được sinh ra trên server và có thể kết nối với nhau.
\end {itemize}

\subsection{AngularJS \& Angular}
\begin{center}
  \captionsetup{type=figure}
  \includegraphics[width=6cm]{img/angular-logo.png}
  \captionof{figure}{Angular}
\end{center}
\newpage
\textbf{AngularJS}

AngularJS hay Angular 1 là một trong những công nghệ Javascript phổ biến nhất trong giới phát triển front-end. Nó được hậu thuẫn bởi Google và một cộng đồng lớn. AngularJS được tạo ra để xây dựng các ứng dụng web động (dynamic web app), và thường được sử dụng để tạo ra các ứng dụng đơn trang (Single Page Application - SPA).

Phiên bản AngularJS được phát triển dựa trên Javascipt nên dễ tiếp cận và được sử dụng rộng rãi. Tuy nhiên về mặt hiệu năng thì AngularJS không được đánh giá cao, thường bị đem ra so sánh với React. Hiện nay, những công ty phát triển phần mềm thường không chọn AngularJS để phát triển một sản phẩm mới.

Những vẫn đề mà Angular có thể giải quyết:
\begin{itemize}
    \item \textbf{Đăng ký callbacks:} Đăng ký hàm gọi lại làm phức tạp code. Loại bỏ mã soạn sẵn phổ biến như gọi lại là một điều tốt. Nó giúp giảm đáng kể số lượng mã hóa JavaScript mà bạn phải thực hiện và giúp dễ dàng xem ứng dụng của bạn làm gì.
    \item \textbf{Thao tác với HTML DOM bằng lập trình:} Thao tác HTML DOM là nền tảng của các ứng dụng AJAX, nhưng nó cồng kềnh và dễ bị lỗi. Bằng cách mô tả khai báo giao diện người dùng sẽ thay đổi như thế nào khi trạng thái ứng dụng của bạn thay đổi, ta sẽ thoát khỏi các tác vụ thao tác DOM cấp thấp. Hầu hết các ứng dụng được viết bằng AngularJS không bao giờ phải thao tác lập trình DOM, mặc dù vẫn có thể thao tác nếu muốn.
    \item \textbf{Sắp xếp dữ liệu và dữ liệu giao diện người dùng:} Các hoạt động CRUD chiếm phần lớn các nhiệm vụ của ứng dụng AJAX. Luồng dữ liệu sắp xếp theo thứ tự từ máy chủ sang đối tượng bên trong sang dạng HTML, cho phép người dùng sửa đổi biểu mẫu, xác thực biểu mẫu, hiển thị lỗi xác thực, quay lại mô hình bên trong và sau đó quay lại máy chủ, tạo ra rất nhiều bản tóm tắt code. AngularJS loại bỏ gần như tất cả các mẫu soạn sẵn này, để lại mã mô tả dòng chảy chung của ứng dụng thay vì tất cả các chi tiết triển khai.
    \item \textbf{Viết hàng tấn mã khởi tạo chỉ để bắt đầu:} Thông thường, ta cần phải viết rất nhiều khai báo khởi tạo chỉ để ứng dụng AJAX cơ bản như "Hello World" hoạt động. Với AngularJS, bạn có thể tự khởi động ứng dụng của mình một cách dễ dàng bằng các dịch vụ, được tự động đưa vào ứng dụng của bạn theo kiểu tiêm phụ thuộc giống như Guice. Điều này cho phép bạn bắt đầu phát triển các tính năng một cách nhanh chóng. Như một phần thêm vào, bạn có toàn quyền kiểm soát quá trình khởi tạo trong các trình kiểm tra tự động.
\end{itemize}

\textbf{Angular JS và Angular:}

Các phiên bản tiếp theo của Angular 2, 4, 5, 6, 7, 8 có tên chính thức là Angular, ra đời với nhiều sự khác biệt và cải tiến so với AngularJS:
\begin {itemize}
\item TypeScript thay cho JavaScript làm ngôn ngữ mặc định
\item Kiến trúc component-based
\item Cải thiện hiệu năng trên nền tảng web và mobile
\end {itemize}

Vì Angular 2 được viết lại từ đầu nên khác biệt hoàn toàn so với AngularJS nên việc nâng cấp từ AngulaJS lên Angular 2 khá khó khăn. Hiện nay cộng đồng lập trình viên vẫn đang sử dụng cả hai framework.

\textbf{Angular:}

Thế mạnh của Angular:
\begin{itemize}
    \item \textbf{Angular trình bày cho bạn không chỉ các công cụ mà còn thiết kế các mẫu để xây dựng dự án của bạn một cách dễ duy trì.} Khi một ứng dụng Angular được tạo ra đúng cách, ta sẽ không gặp phải một mớ các lớp và phương thức khó sửa đổi và thậm chí khó kiểm tra hơn. Code được cấu trúc thuận tiện và không cần phải mất nhiều thời gian để hiểu những gì đang diễn ra.
    \item \textbf{Nó là JavaScript, nhưng tốt hơn.} Angular được xây dựng với TypeScript, lần lượt dựa vào JS ES6. Ta không cần học một ngôn ngữ hoàn toàn mới, nhưng ta vẫn nhận được các tính năng tốt hơn.
    \item \textbf{Không cần phải tạo lại bicycle.} Với Angular, ta đã có rất nhiều công cụ để bắt đầu tạo ứng dụng ngay lập tức. Ta có các chỉ thị để cung cấp cho các phần tử HTML hành vi động. Ta có thể tăng sức mạnh cho các biểu mẫu bằng FormControl và giới thiệu các quy tắc xác thực khác nhau. Ta có thể dễ dàng gửi các yêu cầu HTTP không đồng bộ thuộc nhiều loại khác nhau. Ta có thể thiết lập định tuyến với ít rắc rối. Và còn nhiều điều tiện ích nữa mà Angular có thể cung cấp!
    \item \textbf{Các thành phần được tách rời.} Angular cố gắng để loại bỏ khớp nối chặt chẽ giữa các thành phần khác nhau của ứng dụng. Injection xảy ra theo kiểu NodeJS và ta có thể dễ dàng thay thế các thành phần khác nhau.
    \item \textbf{Tất cả các thao tác DOM diễn ra ở nơi nó nên diễn ra.} Với Angular, ta không kết hợp chặt chẽ việc trình bày và luận lí của ứng dụng làm cho việc đánh dấu của ta đơn giản và đơn giản hơn nhiều.
    \item \textbf{Kiểm thử là chủ yếu.} Angular có nghĩa là được kiểm thử kỹ lưỡng và nó hỗ trợ cả kiểm thử đơn vị và đầu cuối với các công cụ như Jasmine và Protractor.
    \item \textbf{Angular tương thích cho cả điện thoại va máy tính,} có nghĩa là ta có một framework cho nhiều nền tảng.
    \item \textbf{Angular được bảo trì tốt} và có một cộng đồng lớn và hệ sinh thái. Bạn có thể tìm thấy nhiều tài liệu trên framework này cũng như nhiều công cụ của bên thứ ba cũng tương đối hữu ích.
\end{itemize}

Như vậy, có thể nói Angular không chỉ là một framework, mà là một nền tảng trao quyền cho các nhà phát triển để xây dựng các ứng dụng cho web, di động và máy tính.

\subsection{VueJS}
\begin{center}
  \captionsetup{type=figure}
  \includegraphics[width=4cm]{img/Vue-logo.png}
  \captionof{figure}{VueJS}
\end{center}

Vue.js gọi tắt là Vue, là một framework linh động (nguyên bản tiếng Anh: progressive framework) dùng để xây dựng giao diện người dùng (user interfaces). Khác với các framework nguyên khối (monolithic), Vue được thiết kế từ đầu theo hướng cho phép và khuyến khích việc phát triển ứng dụng theo từng bước. Khi phát triển lớp giao diện (view layer), người dùng chỉ cần dùng thư viện lõi (core library) của Vue, vốn rất dễ học và tích hợp với các thư viện hoặc dự án có sẵn. Cùng lúc đó, nếu kết hợp với những kĩ thuật hiện đại như SFC (single file components) và các thư viện hỗ trợ, Vue cũng đáp ứng được dễ dàng nhu cầu xây dựng những ứng dụng đơn trang với độ phức tạp cao hơn nhiều.

Vue có nhiều điểm tương đồng với React với Virtual DOM và các component có thể tái sử dụng. Vue.js cũng hỗ trợ tích hợp những thư viện khác vào framework mà không cần tốn quá nhiều công sức.
\subsection{Backbone.js}
\begin{center}
  \captionsetup{type=figure}
  \includegraphics[width=8cm]{img/backbone.png}
  \captionof{figure}{Backbone}
\end{center}

Backbone là một thư viện Javascript, phục vụ cho việc phát triển front-end. Backbone sử dụng các component Event, Model, Collection, View, Router để tạo nên ứng dụng web. Backbone được các lập trình viên dùng nhiều bởi nó dễ sử dụng và áp dụng cho các ứng dụng Javascript.

Backbone đơn giản, gọn nhẹ và có thể kết hợp cùng các thư viện khác. Các ứng dụng thực tế sử dụng Backbone như LinkedIn Mobile, Foursquare, Basecamp,...

Backbone không có controller nên model và view được đồng bộ rất tốt. Tuy nhiên, việc không có controller đôi khi có thể gây ra hiện tượng nhập nhằn vì các thao tác làm trong controler giờ được trải rải rác vào model và view.
\subsection{Ember.js}
\begin{center}
  \captionsetup{type=figure}
  \includegraphics[width=4cm]{img/ember-logo.png}
  \captionof{figure}{Ember.js}
\end{center}

Ember là một front-end framework Javascript mã nguồn mở vận hành trên mô hình Model-View-Viewmodel (MVVM). Ember cho phép phát triển tạo ra các ứng dụng đơn trang có thể mở rộng bằng cách kết hợp các thành ngữ phổ biến và các thực tiễn tốt nhất vào khung.
\subsection{Kết luận và lựa chọn}
Hiện nay có rất nhiều thư viện và frame-work hỗ trợ việc xây dựng giao diện, sau khi tìm hiểu nhóm quyết định lựa chọn ReactJs là công cụ phát triển chính dựa vào những đặc điểm nổi bật sau:
\begin{itemize}
    \item React xây dựng các ứng dụng đơn trang với hiệu suất cao, đã được kiểm chứng như Facebook, Instagram.
    \item React là công nghệ mã nguồn mở, có nhiều thư viện và các công cụ hỗ trợ xây dựng ứng dụng bằng React như Flux, Redux,...
    \item Hiện nay React đang được Facebook hỗ trợ và có cộng đồng sử dụng lớn, vì vậy React là một công nghệ đáng tin cậy và được cập nhập liên tục.
    \item Việc xây dựng ứng dụng di động trở nên dễ dàng hơn với React Native - mobile-framwork của Facebook được xây dựng dựa trên React.
\end {itemize}    