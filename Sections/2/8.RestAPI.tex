\section{REST API}
\subsection{Giới thiệu REST API:}
REST (REpresentational State Transfer) là một tiêu chuẩn thiết kế API (API architectural style) cho các ứng dụng. REST được giới thiệu lần đầu tiên bởi Roy Fielding vào năm 2000 trong luận án nổi tiếng của ông ``Architectural Styles and
the Design of Network-based Software Architecture'' đại học California.

Mặc dù REST có thể sử dụng trên hầu hết mọi giao thức ứng dụng phần mềm, nhưng REST thường biết đến rộng rãi trong các giao thức của ứng dụng Web vì tận dụng được những lợi thế của HTTP, giúp nhà phát triển không cần cài đặt thêm phần mềm hoặc thư viện bổ sung để sử dụng REST API.

\subsection{Một số đặc điểm cơ bản của REST:}
\begin{itemize}
    \item \textbf{Uniform interface:} Các API được thiết kế một cách nhất quán theo các nguyên tắc chung. Ví dụ luôn luôn sử dụng danh từ số nhiều thay vì khi số nhiều, khi số ít, sử dụng dấu gạch ngang để phân tách giữa các từ. Tài nguyên của hệ thống phải tuân theo các quy tắt đặt tên, định dạng dữ liệu (JSON hay XML), được truy cập thông qua một cách tiếp cận phổ biến như HTTP GET và được sửa đổi thông qua các phương thức tương tự.
    \item \textbf{Client-Server:} Client và Server không cần phụ thuộc vào nhau, có thể phát triển độc lập, thậm chí sửa đổi, thay thế miễn sao interface giao tiếp giữa chúng vẫn được giữ nguyên.
    \item \textbf{Stateless}: Server không lưu trữ bất kì thông tin gì về yêu cầu của client, nó sẽ coi mọi yêu cầu là mới. Không section, không history. Vì vậy, mỗi yêu cầu từ client tới server phải chứa tất cả các thông tin cần thiết để  server hiểu được yêu cầu đó.
    \item \textbf{Code on demand:} Sử dụng HTTP status code khi có thể
\end{itemize}


\subsection{Những lợi ích khi sử dụng REST API:}

- Nhờ tính độc lập giữa client và server, nhà phát triển có thể dễ dàng phát triển, sửa đổi thậm chí thay thế client và server miễn sao vẫn giữ nguyên interface giao tiếp giữa chúng.

- API REST luôn độc lập với loại nền tảng hoặc ngôn ngữ: API REST luôn thích nghi với loại cú pháp hoặc nền tảng đang được sử dụng, điều này mang lại sự tự do đáng kể khi thay đổi hoặc kiểm tra môi trường mới trong quá trình phát triển. Với API REST, bạn có thể có máy chủ với các ngôn ngữ như PHP, Java, Python hoặc Node.js.

- Tính nhất quán: Các API được thiết kế một cách nhất quán giúp cho việc phát triển, bảo trì đơn giản hơn, đặc biệt phù hợp với các hệ thống có nhiều module.