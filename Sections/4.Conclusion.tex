\section{Kết quả đạt được}
Trong thời gian thực hiện đề cương nhóm đã đạt được một số kết quả sau:
\begin{itemize}
    \item Tìm hiểu cơ sở lý thuyết về các công nghệ liên quan: Mô hình MVC; ứng dụng đơn trang, phân trang; các công nghệ cho front-end, back-end; cơ sở dữ liệu.
    \begin{itemize}
        \item Lựa chọn hệ cơ sở dữ liệu Oracle cho các dữ liệu có cấu trúc cố định, những dữ liệu có cấu trúc thay đổi thì được lưu vào MongoDB. Ngoài ra nhóm còn lựa chọn Redis làm cache cho hệ thống.
        \item Lựa chọn nền tảng font-end là React kết hợp với Redux, cũng như chọn NodeJs là nền tảng back-end.
    \end{itemize}
    \item Phân tích yêu cầu dự án: nắm bắt mục tiêu, phạm vi đề tài, các chức năng cần có của hệ thống, nghiệp vụ phòng Tổ chức hành chính. Từ đó đưa ra quyết định phù hợp cho hệ thống trong tương lai.
    \item Tìm hiểu các hệ thống tương tự, để làm quen trước công việc xây dựng trong tương lai
\end{itemize}
\section{Thuận lợi}
Sau thời gian nghiên cứu, tìm hiểu đề tài nhóm nhận thấy đề tài có những thuận lợi sau:
\begin{itemize}
    \item Đây là một đề tài được áp dụng vào thực tế cho một phòng ban của trường Đại học Bách Khoa Tp.HCM. Đây là vừa là cơ hội, vừa là thử thách đòi hỏi hệ thống phải hoạt động tốt.
    \item Kinh nghiệm từ đè tài có thể áp dụng vào công việc trong tương lai
    \item Thực hiện đề tài dưới hình thức làm việc nhóm giúp tăng cao được khả năng làm việc nhóm
    \item Nhận được sự chia sẻ, hướng dẫn tận tình từ Th.S Nguyễn Thanh Tùng
\end{itemize}
\section{Khó khăn}
Bên cạnh những thuận lợi nêu trên thì đề tài cũng gặp rất nhiều khó khăn:
\begin{itemize}
    \item Khối lượng dữ liệu lớn, thay đổi liên tục.
    \item Só lượng nghiệp vụ lớn, đa dạng, nhiều nghiệp vụ phức tạp.
    \item Thường xuyên xuất, nhập báo cáo đòi hỏi tốc độ xử lý dữ liệu nhanh.
\end{itemize}
\section{Hướng phát triển trong luận văn}
Hệ thống website Tổ chức hành chính sẽ được đưa vào sử dụng thực tế cho các công việc, nghiệp vụ của phòng Tổ chức hành chính trường Đại học Bách Khoa Tp.HCM.

Cụ thể hệ thống quản lý cán bộ công nhân viên, thống kê, tổng kết sẽ được triển khai và thay thế những công việc nhập liệu thủ công hiện nay của phòng giúp công việc thuận tiện, dễ dàng, giảm bớt được khối lượng công việc, tránh sai sót.

Dữ liệu của phòng sẽ được lưu trữ tập trung tại database, được thiết kế và quản lý dữ liệu một cách khoa học để không bị trùng lặp dữ liệu.

Sau thời gian chạy thử nghiệm tại phòng Tổ chức hành chính, nhóm đã thực hiện lắng nghe các ý kiến đóng góp từ các nhân viên, cán bộ của phòng ban và thực hiện các điều chỉnh phù hợp nhằm tạo ra hệ thống 