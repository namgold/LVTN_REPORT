\section{Giới thiệu phòng tổ chức hành chính của Đại học Bách Khoa - ĐHQG-HCM}
Ngày nay, trong đời sống xã hội nói chung, các cơ quan quản lý nhà nước và các doanh nghiệp sản xuất kinh doanh nói riêng, con người là một nhân tố cực kỳ quan trọng: bằng sự lao động sáng tạo của mình sẽ thúc đẩy mọi sự phát triển của xã hội. Vì vậy đối với bất kỳ lĩnh vực nào thì con người cũng là trung tâm của mọi sự điều khiển.

Quản lý nhân sự là một trong những bộ phận quan trọng trong tổ chức, đặc biệt là trong các tổ chức lớn trong nước và các tổ chức nước ngoài. Sự thành bại của tổ chức phụ thuộc vào cách thức tổ chức nhân sự có tốt hay không. Trong những năm vừa qua quản lý nhân sự đang dần phát triển mạnh mẽ không những ở các tổ chức nước ngoài mà các tổ chức trong nước cũng đang dần nhận thấy sự quan trọng của tổ chức nhân sự trong tổ chức.

Phòng tổ chức hành chính là một phòng ban quan trọng của Đại học Bách Khoa - ĐHQG-HCM. Phòng có chức năng tham mưu, giúp cho Hiệu trưởng và Đảng Ủy trong việc chỉ đạo, thực hiện công tác tổ chức bộ máy quản lý cán bộ nhà trường theo đúng các chủ trương chính sách của Đảng, Nhà nước và các quy định quy chế của Bộ, Đại học Quốc gia và Nhà trường đã ban hành. Tổ chức giao dịch hành chính, trao đổi thông tin giữa Ban Giám hiệu với các cơ quan khác trong nước và giữa Ban Giám hiệu với các đơn vị, CBCNV, sinh viên trong trường.\\

Các chức năng, nhiệm vụ của phòng tổ chức hành chính:
\begin{itemize}
    \item Nghiên cứu, đề xuất xây dựng bộ máy tổ chức đội ngũ và tổ chức điều hành trong trường.
    \item Xây dựng và hướng dẫn thực hiện các nội quy, quy chế về định biên và quản lý biên chế.
    \item Lập và quản lý hồ sơ về lương, thủ tục đề nghị nâng bậc và điều chỉnh lương hàng năm.
    \item Chỉ đạo và thực hiện công tác bảo vệ chính trị, trật tự an ninh trong khuôn viên trường.
    \item Tổ chức quản lý, lưu trữ hồ sơ lý lịch của CBCNV, bổ sung và nhận xét hàng năm.
\end{itemize}

\section{Các vấn đề hiện tại của phòng tổ chức hành chính}
Phòng tổ chức hành chính Đại học Bách Khoa - Đại học Quốc gia TP.HCM hiện tại đang lưu trữ và sử dụng dữ liệu dưới dạng các file Microsoft Access, Microsoft Excel, Microsoft Word và một số tài liệu giấy. Cách thức hoạt động này được triển khai từ nhiều năm trước với những ưu điểm như sau:
\begin{itemize}
    \item Vì sử dụng các phần mềm Microsoft Word, Excel nên thân thiện, dễ sử dụng với cán bộ, nhân viên của phòng Tổ chức hành chính.
    \item Dễ mở rộng nếu có thêm các nghiệp vụ mới.
    \item Dễ tạo báo cáo, thống kê bằng Microsoft Word, Microsoft Excel.
    \item Nhân viên mới dễ dàng làm quen và sử dụng hệ thống.
\end{itemize}
Tuy nhiên, cách hoạt động này chỉ tỏ ra hiệu quả khi hệ thống còn ít người dùng, dữ liệu còn nhỏ, nghiệp vụ ít, đơn giản. Ở thời điểm hiện tại, khi lượng dữ liệu và các nghiệp vụ mà tổ chức hành chính cần quản lý ngày càng nhiều, thì cách thức hoạt động này thể hiện những hạn chế như sau: 
\begin{itemize}
    \item Các nghiệp vụ như nhập liệu, tính toán, lưu trữ hiện nay được cán bộ phòng Tổ chức hành chính thực hiện thủ công nên tốn nhiều thời gian và có thể dẫn đến những sai sót, trùng lặp, thiếu sót.
    \item Dữ liệu ngày càng nhiều, dẫn đến khó khăn trong việc lưu trữ, quản lý, truy xuất.
    \item Không có cơ chế phân quyền cho các cán bộ của phòng, vì vậy có thể dẫn đến các thao tác, hành động vượt quá quyền hạn, ảnh hưởng đến hệ thống. 
    \item Dữ liệu trên website http://www.tchc.hcmut.edu.vn lỗi thời, việc cập nhập dữ liệu lên website khá khó khăn và tốn thời gian.
    \item Cần phải cài đặt các phần mềm mới truy cập được vào hệ thống. Không truy cập được khi sử dụng các thiết bị di động.
    \item Việc trao đổi thông tin trong nội bộ phòng ban và giữa phòng ban với bên ngoài gặp nhiều khó khăn.
    \item Nhân viên mới phải điền thông tin vào các mẫu lý lịch có sẵn và được nhân viên nhập liệu vào hệ thống một cách thủ công, gây tốn thời gian và có thể dẫn đến sai sót thông tin.
    \item Tốn nhiều chi phí cho việc lưu trữ, sắp xếp, tìm kiếm hồ sơ nhân viên trong kho chứa.
    \item Dễ dẫn đến sai sót khi tính toán, đặc biệt là tính toán tiền lương, điểm thi đua cán bộ... Nếu có sai sót thì sẽ dẫn đến hậu quả nghiêm trọng.
\end{itemize}
\textbf{Kết luận chung:}

Hệ thống phòng Tổ chức hành chính của trường Đại học Bách Khoa - Đại học Quốc gia TP.HCM hiện tại đã lỗi thời, gây nhiều khó khăn cho cán bộ, nhân viên nhà trường trong quá trình vận hành và sử dụng. Vì vậy, việc xây dựng hệ thống Tổ chức hành chính mới để giải quyết được các các khó khăn hiện tại, bổ sung các tính năng mới, cải thiện công tác quản lý, vận hành công việc của phòng Tổ chức hành chính là thực sự cần thiết.
    
\section{Mục tiêu của đề tài}
Mục tiêu của đề tài này là xây dựng một hệ thống (website, ứng dụng) phòng tổ chức hành chính của Đại học Bách Khoa - Đại học Quốc gia TP.HCM với các tính năng như sau:

\begin{itemize}
    \item Quản lý thông tin cán bộ công nhân viên nhà trường, bao gồm các tính năng tìm kiếm, thêm, xoá, sửa, lưu thông tin, nhập dữ liệu từ file Excel,...
    \item Website truyền thông với giao diện đẹp, dữ liệu đầy đủ, update thường xuyên, cung cấp các thông tin mới, quan trọng với cán bộ, sinh viên, người dùng,... 
    \item Tính năng thống kê, tổng hợp, tính điểm thi đua.
    \item Tính năng thống kê, tổng hợp, tính điểm điểm thưởng.
    \item Quản lý bảo hiểm cho cán bộ.
    \item Quản lý các cấp, đơn vị hành chính của nhà trường.
    \item Quản lý quy trình khen thưởng, kỷ luật.
    \item Quản lý lương cán bộ.
    \item Quản lý chức danh, chức vụ của cán bộ.
    \item Quản lý quá trình đào tạo chuyên môn.
    \item Quản lý quá trình công tác trong nước và nước ngoài của cán bộ.
    \item Bồi dưỡng nghiệp vụ cho cán bộ.
    \item Quản lý cán bộ công tác trong và ngoài nước.
    \item Tuyển chọn cán bộ.
    \item Chia sẻ thông tin cho các cán bộ, phòng ban khác.
    \item Quản lý lương, thưởng và các kế hoạch bảo hiểm, trợ cấp của người lao động.
    \item Xuất các báo cáo, thống kê theo yêu cầu và theo định kỳ.
\end{itemize}
\section{Phạm vi đề tài}
Phạm vi các đối tượng mà đề tài hướng đến là:
\begin{itemize}
    \item Cán bộ, nhân viên phòng Tổ chức hành chính với vai trò vận hành quản lý hệ thống.
    \item Cán bộ, công nhân viên trường Đại học Bách Khoa - Đại học Quốc gia TP.HCM.
    \item Sinh viên và cựu sinh viên trường Đại học Bách Khoa - Đại học Quốc gia TP.HCM.
    \item Các đơn vị, doanh nghiệp là đối tác của nhà trường.
\end{itemize}
\section{Khó khăn và thử thách}
Các khó khăn, thử thách mà đề tài gặp phải:
\begin{itemize}
    \item Độ ổn định của hệ thống.
    \item Lượng cán bộ sử dụng lớn.
    \item Số lượng nghiệp vụ lớn.
    \item Đầy đủ các chức năng nhưng vẫn phải đảm bảo dễ dàng sử dụng.
\end{itemize}